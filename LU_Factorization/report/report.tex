\documentclass{article}
\usepackage[top= 15mm, bottom = 25mm]{geometry}
\usepackage{amsfonts,latexsym}
\newcommand{\fl}{{\mbox{fl}}}
\newcommand{\bx}{{\mathbf x}}
\newcommand{\by}{{\mathbf y}}
\newcommand{\bb}{{\mathbf b}}

\setlength{\parindent}{4mm}
\setlength{\parskip}{0.3em}


\title{MATH226 Project I \\ LU Factorization}
\author{Erdal Sidal Dogan \\ MEF University \\ \#041702023}
\begin{document}
	\maketitle
	\section{Project Definition \& Goal}
	Main goal of the project is to design and implement a computer program which will solve a given system of linear equation by using \textit{LU Decomposition} method. LU Decomposition, often referred as LU factorization, is a method where a given matrix $A$ is represented as product of two other matrices L and U. L is Lower Triangular matrix where the main diagonal is all 1's and there are zeros only above the main diagonal. Likewise, U is the Upper Triangular Matrix where there are not any values except zero below the main diagonal. 
	\begin{equation} \label{alu}
 	  A = LU 
 \end{equation} 
\begin{equation}
A=\left[\begin{array}{ccccc} 
a_{11} & a_{12} & a_{1,3} \\
a_{21} & a_{22} & a_{2,3} \\
a_{31} & a_{32} & a_{33} 
\end{array}\right] = \left[\begin{array}{ccccc} 
l_{11} & 0 & 0 \\
l_{21} & l_{22} & 0 \\
l_{31} & l_{32} & l_{33} 
\end{array}\right]\left[\begin{array}{ccccc} 
u_{11} & u_{12} & u_{13} \\
0 & u_{22} & u_{23} \\
0 & 0 & u_{33} 
\end{array}\right] 
\end{equation}\\\\
This factorization is called the \emph{LU factorization} of $A$.\\ \\We also know that a linear systems of equations are defined as;
 	\begin{equation} \label{axb}
 	  Ax = b 
 \end{equation}
Given the LU factorization of the matrix $A$, we can solve the linear system 
(\ref{axb}}) in two steps: substitute (\ref{alu}}}) into (\ref{axb}})
to obtain
\[
LU{\bx}={\bb},
\]
and then solve the triangular systems, in order,
\begin{eqnarray}
\label{fsubst}
L{\by}&=&{\bb}, \\
\label{bsubst}
U{\bx}&=&{\by}.
\end{eqnarray}

Equations such as (\ref{fsubst}) and (\ref{bsubst}) can be solved with methods which are called \textit{Forward Substitution} and \textit{Backward Substitution}. Which are computationally cheap also.
\subsection{Forward Substitution}
For lower triangular system $L{\bx}={\bb}$
\begin{equation}
  x_1 = b_1/l_{11},\hspace{4mm} x_i = \left( b_i - \sum_{j=1}^{i-1}l_{ij}x_i\right) \mathbin{/} l_{ii}, \hspace{4mm} i =2,\ldots,n
\end{equation}
\subsubsection{Psuedocode}

\subsection{Backward Substitution}
For upper triangular system $U{\bx}={\by}$
\begin{equation}
  x_n = b_n/u_{nn},\hspace{4mm} x_i = \left( b_i - \sum_{j=i+1}^{n}u_{ij}x_j\right) \mathbin{/} u_{ii}, \hspace{4mm} i =n -1,\ldots,1
\end{equation}

LU Decomposition method is computationally cheaper for solving a system comparing to \textit{Gaussian Elimination} especially on large-dimension matrices. 	\textit{Java Language} is utilized for this project. \\
	\section{Implementation \& Algorithm}
	
Not every matrix has LU Decomposition. If there are any zeros at pivot locations of a matrix, its LU factorization cannot be found without making any changes on the original matrix. In order to address this problem, we use \textit{row interchanges}. If there are any zeros in pivot positions in a matrix, the program will automatically use row interchanging method on it. This process of interchanging rows are called \textit{Partial Pivoting}.

Of course, there are matrices that has \textit{LU Fact.} as it is. For this matrices, we use an algorithm called \textit{Doolittle}, for others, i.e. matrices with zeros in pivot positions; first we convert them to regular matrices by applying \textit{Partial Pivoting}, then find the LU Fact.
\subsection{Doolittle Algorithm}
	Doolittle is a simple and straightforward algorithm to compute \textit{Lower Triangular (L)} and \textit{Upper Triangular(U)} matrix. If anyone writes down the A, L and U matrices and starts to compute values in the lower and upper triangular matrices can see the pattern of the solution. 

	
	
\end{document}